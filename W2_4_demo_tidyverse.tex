% Options for packages loaded elsewhere
\PassOptionsToPackage{unicode}{hyperref}
\PassOptionsToPackage{hyphens}{url}
%
\documentclass[
]{article}
\usepackage{lmodern}
\usepackage{amssymb,amsmath}
\usepackage{ifxetex,ifluatex}
\ifnum 0\ifxetex 1\fi\ifluatex 1\fi=0 % if pdftex
  \usepackage[T1]{fontenc}
  \usepackage[utf8]{inputenc}
  \usepackage{textcomp} % provide euro and other symbols
\else % if luatex or xetex
  \usepackage{unicode-math}
  \defaultfontfeatures{Scale=MatchLowercase}
  \defaultfontfeatures[\rmfamily]{Ligatures=TeX,Scale=1}
\fi
% Use upquote if available, for straight quotes in verbatim environments
\IfFileExists{upquote.sty}{\usepackage{upquote}}{}
\IfFileExists{microtype.sty}{% use microtype if available
  \usepackage[]{microtype}
  \UseMicrotypeSet[protrusion]{basicmath} % disable protrusion for tt fonts
}{}
\makeatletter
\@ifundefined{KOMAClassName}{% if non-KOMA class
  \IfFileExists{parskip.sty}{%
    \usepackage{parskip}
  }{% else
    \setlength{\parindent}{0pt}
    \setlength{\parskip}{6pt plus 2pt minus 1pt}}
}{% if KOMA class
  \KOMAoptions{parskip=half}}
\makeatother
\usepackage{xcolor}
\IfFileExists{xurl.sty}{\usepackage{xurl}}{} % add URL line breaks if available
\IfFileExists{bookmark.sty}{\usepackage{bookmark}}{\usepackage{hyperref}}
\hypersetup{
  hidelinks,
  pdfcreator={LaTeX via pandoc}}
\urlstyle{same} % disable monospaced font for URLs
\usepackage[margin=1in]{geometry}
\usepackage{color}
\usepackage{fancyvrb}
\newcommand{\VerbBar}{|}
\newcommand{\VERB}{\Verb[commandchars=\\\{\}]}
\DefineVerbatimEnvironment{Highlighting}{Verbatim}{commandchars=\\\{\}}
% Add ',fontsize=\small' for more characters per line
\usepackage{framed}
\definecolor{shadecolor}{RGB}{248,248,248}
\newenvironment{Shaded}{\begin{snugshade}}{\end{snugshade}}
\newcommand{\AlertTok}[1]{\textcolor[rgb]{0.94,0.16,0.16}{#1}}
\newcommand{\AnnotationTok}[1]{\textcolor[rgb]{0.56,0.35,0.01}{\textbf{\textit{#1}}}}
\newcommand{\AttributeTok}[1]{\textcolor[rgb]{0.77,0.63,0.00}{#1}}
\newcommand{\BaseNTok}[1]{\textcolor[rgb]{0.00,0.00,0.81}{#1}}
\newcommand{\BuiltInTok}[1]{#1}
\newcommand{\CharTok}[1]{\textcolor[rgb]{0.31,0.60,0.02}{#1}}
\newcommand{\CommentTok}[1]{\textcolor[rgb]{0.56,0.35,0.01}{\textit{#1}}}
\newcommand{\CommentVarTok}[1]{\textcolor[rgb]{0.56,0.35,0.01}{\textbf{\textit{#1}}}}
\newcommand{\ConstantTok}[1]{\textcolor[rgb]{0.00,0.00,0.00}{#1}}
\newcommand{\ControlFlowTok}[1]{\textcolor[rgb]{0.13,0.29,0.53}{\textbf{#1}}}
\newcommand{\DataTypeTok}[1]{\textcolor[rgb]{0.13,0.29,0.53}{#1}}
\newcommand{\DecValTok}[1]{\textcolor[rgb]{0.00,0.00,0.81}{#1}}
\newcommand{\DocumentationTok}[1]{\textcolor[rgb]{0.56,0.35,0.01}{\textbf{\textit{#1}}}}
\newcommand{\ErrorTok}[1]{\textcolor[rgb]{0.64,0.00,0.00}{\textbf{#1}}}
\newcommand{\ExtensionTok}[1]{#1}
\newcommand{\FloatTok}[1]{\textcolor[rgb]{0.00,0.00,0.81}{#1}}
\newcommand{\FunctionTok}[1]{\textcolor[rgb]{0.00,0.00,0.00}{#1}}
\newcommand{\ImportTok}[1]{#1}
\newcommand{\InformationTok}[1]{\textcolor[rgb]{0.56,0.35,0.01}{\textbf{\textit{#1}}}}
\newcommand{\KeywordTok}[1]{\textcolor[rgb]{0.13,0.29,0.53}{\textbf{#1}}}
\newcommand{\NormalTok}[1]{#1}
\newcommand{\OperatorTok}[1]{\textcolor[rgb]{0.81,0.36,0.00}{\textbf{#1}}}
\newcommand{\OtherTok}[1]{\textcolor[rgb]{0.56,0.35,0.01}{#1}}
\newcommand{\PreprocessorTok}[1]{\textcolor[rgb]{0.56,0.35,0.01}{\textit{#1}}}
\newcommand{\RegionMarkerTok}[1]{#1}
\newcommand{\SpecialCharTok}[1]{\textcolor[rgb]{0.00,0.00,0.00}{#1}}
\newcommand{\SpecialStringTok}[1]{\textcolor[rgb]{0.31,0.60,0.02}{#1}}
\newcommand{\StringTok}[1]{\textcolor[rgb]{0.31,0.60,0.02}{#1}}
\newcommand{\VariableTok}[1]{\textcolor[rgb]{0.00,0.00,0.00}{#1}}
\newcommand{\VerbatimStringTok}[1]{\textcolor[rgb]{0.31,0.60,0.02}{#1}}
\newcommand{\WarningTok}[1]{\textcolor[rgb]{0.56,0.35,0.01}{\textbf{\textit{#1}}}}
\usepackage{graphicx,grffile}
\makeatletter
\def\maxwidth{\ifdim\Gin@nat@width>\linewidth\linewidth\else\Gin@nat@width\fi}
\def\maxheight{\ifdim\Gin@nat@height>\textheight\textheight\else\Gin@nat@height\fi}
\makeatother
% Scale images if necessary, so that they will not overflow the page
% margins by default, and it is still possible to overwrite the defaults
% using explicit options in \includegraphics[width, height, ...]{}
\setkeys{Gin}{width=\maxwidth,height=\maxheight,keepaspectratio}
% Set default figure placement to htbp
\makeatletter
\def\fps@figure{htbp}
\makeatother
\setlength{\emergencystretch}{3em} % prevent overfull lines
\providecommand{\tightlist}{%
  \setlength{\itemsep}{0pt}\setlength{\parskip}{0pt}}
\setcounter{secnumdepth}{-\maxdimen} % remove section numbering

\author{}
\date{\vspace{-2.5em}}

\begin{document}

\hypertarget{demo-tidyverse}{%
\subsection{Demo Tidyverse}\label{demo-tidyverse}}

\href{W2_4_demo_tidyverse.Rmd}{Download the code as an Rmd-File}

Depending on your knowledge of \texttt{R}, getting an overview of the
data we imported last week might have been quite a challenge.
Surprisingly enough, importing, cleaning and exploring your data can be
the most challenging, time consuming part of a project. RStudio and the
tidyverse offer many helpful tools to make this part easier (and more
fun). You have read chapters on \texttt{dplyr} and \texttt{magrittr} as
a preparation for this exercise. Before we start with the exercise
however, this demo illustrates a simple approach offered by tidyverse
which is applicable to sf-objects.

Assume we want to calculate the timelag between subsequent positions. To
achieve this we can use the function \texttt{difftime()} combined with
\texttt{lead()} from \texttt{dplyr}. Let's look at these functions one
by one.

\hypertarget{difftime}{%
\subsubsection{\texorpdfstring{\texttt{difftime}}{difftime}}\label{difftime}}

\texttt{difftime} takes two \texttt{POSIXct} values.

\begin{Shaded}
\begin{Highlighting}[]
\NormalTok{now <-}\StringTok{ }\KeywordTok{Sys.time}\NormalTok{()}

\NormalTok{later <-}\StringTok{ }\NormalTok{now }\OperatorTok{+}\StringTok{ }\DecValTok{10000}

\NormalTok{time_difference <-}\StringTok{ }\KeywordTok{difftime}\NormalTok{(later,now)}
\end{Highlighting}
\end{Shaded}

\begin{Shaded}
\begin{Highlighting}[]
\NormalTok{time_difference}
\end{Highlighting}
\end{Shaded}

\begin{verbatim}
## Time difference of 2.777778 hours
\end{verbatim}

You can also specify the unit of the output.

\begin{Shaded}
\begin{Highlighting}[]
\NormalTok{time_difference <-}\StringTok{ }\KeywordTok{difftime}\NormalTok{(later,now,}\DataTypeTok{units =} \StringTok{"mins"}\NormalTok{)}
\end{Highlighting}
\end{Shaded}

\begin{Shaded}
\begin{Highlighting}[]
\NormalTok{time_difference}
\end{Highlighting}
\end{Shaded}

\begin{verbatim}
## Time difference of 166.6667 mins
\end{verbatim}

\texttt{difftime} returns an object of the Class \texttt{difftime}.
However in our case, numeric values would be more handy than the Class
\texttt{difftime}. So we'll wrap the command in \texttt{as.numeric()}:

\begin{Shaded}
\begin{Highlighting}[]
\KeywordTok{str}\NormalTok{(time_difference)}
\end{Highlighting}
\end{Shaded}

\begin{verbatim}
##  'difftime' num 166.666666666667
##  - attr(*, "units")= chr "mins"
\end{verbatim}

\begin{Shaded}
\begin{Highlighting}[]
\NormalTok{time_difference <-}\StringTok{ }\KeywordTok{as.numeric}\NormalTok{(}\KeywordTok{difftime}\NormalTok{(later,now,}\DataTypeTok{units =} \StringTok{"mins"}\NormalTok{))}

\KeywordTok{str}\NormalTok{(time_difference)}
\end{Highlighting}
\end{Shaded}

\begin{verbatim}
##  num 167
\end{verbatim}

\hypertarget{lead-lag}{%
\subsubsection{\texorpdfstring{\texttt{lead()} /
\texttt{lag()}}{lead() / lag()}}\label{lead-lag}}

\texttt{lead()} and \texttt{lag()} return a vector of the same length as
the input, just offset by a specific number of values (default is 1).
Consider the following sequence:

\begin{Shaded}
\begin{Highlighting}[]
\NormalTok{numbers <-}\StringTok{ }\DecValTok{1}\OperatorTok{:}\DecValTok{10}

\NormalTok{numbers}
\end{Highlighting}
\end{Shaded}

\begin{verbatim}
##  [1]  1  2  3  4  5  6  7  8  9 10
\end{verbatim}

We can now run \texttt{lead()} and \texttt{lag()} on this sequence to
illustrate the output. \texttt{n\ =} specifies the offset,
\texttt{default\ =} specifies the default value used to ``fill'' the
emerging ``empty spaces'' of the vector. This helps us performing
operations on subsequent values in a vector (or rows in a table).

\begin{Shaded}
\begin{Highlighting}[]
\KeywordTok{library}\NormalTok{(dplyr)}
\end{Highlighting}
\end{Shaded}

\begin{verbatim}
## Warning: package 'dplyr' was built under R version 4.0.5
\end{verbatim}

\begin{verbatim}
## 
## Attaching package: 'dplyr'
\end{verbatim}

\begin{verbatim}
## The following objects are masked from 'package:stats':
## 
##     filter, lag
\end{verbatim}

\begin{verbatim}
## The following objects are masked from 'package:base':
## 
##     intersect, setdiff, setequal, union
\end{verbatim}

\begin{Shaded}
\begin{Highlighting}[]
\KeywordTok{lead}\NormalTok{(numbers)}
\end{Highlighting}
\end{Shaded}

\begin{verbatim}
##  [1]  2  3  4  5  6  7  8  9 10 NA
\end{verbatim}

\begin{Shaded}
\begin{Highlighting}[]
\KeywordTok{lead}\NormalTok{(numbers,}\DataTypeTok{n =} \DecValTok{2}\NormalTok{)}
\end{Highlighting}
\end{Shaded}

\begin{verbatim}
##  [1]  3  4  5  6  7  8  9 10 NA NA
\end{verbatim}

\begin{Shaded}
\begin{Highlighting}[]
\KeywordTok{lag}\NormalTok{(numbers)}
\end{Highlighting}
\end{Shaded}

\begin{verbatim}
##  [1] NA  1  2  3  4  5  6  7  8  9
\end{verbatim}

\begin{Shaded}
\begin{Highlighting}[]
\KeywordTok{lag}\NormalTok{(numbers,}\DataTypeTok{n =} \DecValTok{5}\NormalTok{)}
\end{Highlighting}
\end{Shaded}

\begin{verbatim}
##  [1] NA NA NA NA NA  1  2  3  4  5
\end{verbatim}

\begin{Shaded}
\begin{Highlighting}[]
\KeywordTok{lag}\NormalTok{(numbers,}\DataTypeTok{n =} \DecValTok{5}\NormalTok{, }\DataTypeTok{default =} \DecValTok{0}\NormalTok{)}
\end{Highlighting}
\end{Shaded}

\begin{verbatim}
##  [1] 0 0 0 0 0 1 2 3 4 5
\end{verbatim}

\hypertarget{mutate}{%
\subsubsection{\texorpdfstring{\texttt{mutate()}}{mutate()}}\label{mutate}}

Using the above functions (\texttt{difftime()} and \texttt{lead()}), we
can calculate the time lag, that is, the time difference between
consecutive positions. We will try this on a dummy version of our
wildboar dataset.

\begin{Shaded}
\begin{Highlighting}[]
\NormalTok{wildschwein <-}\StringTok{ }\KeywordTok{tibble}\NormalTok{(}
  \DataTypeTok{TierID =} \KeywordTok{c}\NormalTok{(}\KeywordTok{rep}\NormalTok{(}\StringTok{"Hans"}\NormalTok{,}\DecValTok{5}\NormalTok{),}\KeywordTok{rep}\NormalTok{(}\StringTok{"Klara"}\NormalTok{,}\DecValTok{5}\NormalTok{)),}
  \DataTypeTok{DatetimeUTC =} \KeywordTok{rep}\NormalTok{(}\KeywordTok{as.POSIXct}\NormalTok{(}\StringTok{"2015-01-01 00:00:00"}\NormalTok{,}\DataTypeTok{tz =} \StringTok{"UTC"}\NormalTok{)}\OperatorTok{+}\DecValTok{0}\OperatorTok{:}\DecValTok{4}\OperatorTok{*}\DecValTok{15}\OperatorTok{*}\DecValTok{60}\NormalTok{, }\DecValTok{2}\NormalTok{)}
\NormalTok{  )}

\NormalTok{wildschwein}
\end{Highlighting}
\end{Shaded}

\begin{verbatim}
## # A tibble: 10 x 2
##    TierID DatetimeUTC        
##    <chr>  <dttm>             
##  1 Hans   2015-01-01 00:00:00
##  2 Hans   2015-01-01 00:15:00
##  3 Hans   2015-01-01 00:30:00
##  4 Hans   2015-01-01 00:45:00
##  5 Hans   2015-01-01 01:00:00
##  6 Klara  2015-01-01 00:00:00
##  7 Klara  2015-01-01 00:15:00
##  8 Klara  2015-01-01 00:30:00
##  9 Klara  2015-01-01 00:45:00
## 10 Klara  2015-01-01 01:00:00
\end{verbatim}

To calculate the \texttt{timelag} with base-R, we need to mention
\texttt{wildschwein} three times

\begin{Shaded}
\begin{Highlighting}[]
\NormalTok{wildschwein}\OperatorTok{$}\NormalTok{timelag  <-}\StringTok{ }\KeywordTok{as.numeric}\NormalTok{(}\KeywordTok{difftime}\NormalTok{(}\KeywordTok{lead}\NormalTok{(wildschwein}\OperatorTok{$}\NormalTok{DatetimeUTC), wildschwein}\OperatorTok{$}\NormalTok{DatetimeUTC))}
\end{Highlighting}
\end{Shaded}

Using \texttt{mutate()} we can simplify this operation slightly:

\begin{Shaded}
\begin{Highlighting}[]
\NormalTok{wildschwein <-}\StringTok{ }\KeywordTok{mutate}\NormalTok{(wildschwein,}\DataTypeTok{timelag =} \KeywordTok{as.numeric}\NormalTok{(}\KeywordTok{difftime}\NormalTok{(}\KeywordTok{lead}\NormalTok{(DatetimeUTC),DatetimeUTC)))}

\NormalTok{wildschwein}
\end{Highlighting}
\end{Shaded}

\begin{verbatim}
## # A tibble: 10 x 3
##    TierID DatetimeUTC         timelag
##    <chr>  <dttm>                <dbl>
##  1 Hans   2015-01-01 00:00:00      15
##  2 Hans   2015-01-01 00:15:00      15
##  3 Hans   2015-01-01 00:30:00      15
##  4 Hans   2015-01-01 00:45:00      15
##  5 Hans   2015-01-01 01:00:00     -60
##  6 Klara  2015-01-01 00:00:00      15
##  7 Klara  2015-01-01 00:15:00      15
##  8 Klara  2015-01-01 00:30:00      15
##  9 Klara  2015-01-01 00:45:00      15
## 10 Klara  2015-01-01 01:00:00      NA
\end{verbatim}

\hypertarget{group_by}{%
\subsubsection{\texorpdfstring{\texttt{group\_by()}}{group\_by()}}\label{group_by}}

You might have noticed that \texttt{timelag} is calculated across
different individuals (\texttt{Hans} and \texttt{Klara}), which does not
make much sense. To avoid this, we need to specify that \texttt{timelag}
should just be calculated between consecutive rows \emph{of the same
individual}. We can implement this by using \texttt{group\_by()}.

\begin{Shaded}
\begin{Highlighting}[]
\NormalTok{wildschwein <-}\StringTok{ }\KeywordTok{group_by}\NormalTok{(wildschwein,TierID)}
\end{Highlighting}
\end{Shaded}

After adding this grouping variable, calculating the \texttt{timelag}
automatically accounts for the individual trajectories.

\begin{Shaded}
\begin{Highlighting}[]
\NormalTok{wildschwein <-}\StringTok{ }\KeywordTok{mutate}\NormalTok{(wildschwein,}\DataTypeTok{timelag =} \KeywordTok{as.numeric}\NormalTok{(}\KeywordTok{difftime}\NormalTok{(}\KeywordTok{lead}\NormalTok{(DatetimeUTC),DatetimeUTC)))}



\NormalTok{wildschwein}
\end{Highlighting}
\end{Shaded}

\begin{verbatim}
## # A tibble: 10 x 3
## # Groups:   TierID [2]
##    TierID DatetimeUTC         timelag
##    <chr>  <dttm>                <dbl>
##  1 Hans   2015-01-01 00:00:00      15
##  2 Hans   2015-01-01 00:15:00      15
##  3 Hans   2015-01-01 00:30:00      15
##  4 Hans   2015-01-01 00:45:00      15
##  5 Hans   2015-01-01 01:00:00      NA
##  6 Klara  2015-01-01 00:00:00      15
##  7 Klara  2015-01-01 00:15:00      15
##  8 Klara  2015-01-01 00:30:00      15
##  9 Klara  2015-01-01 00:45:00      15
## 10 Klara  2015-01-01 01:00:00      NA
\end{verbatim}

\hypertarget{summarise}{%
\subsubsection{\texorpdfstring{\texttt{summarise()}}{summarise()}}\label{summarise}}

If we want to summarise our data and get metrics \emph{per animal}, we
can use the \texttt{dplyr} function \texttt{summarise()}. In contrast to
\texttt{mutate()}, which just adds a new column to the dataset,
\texttt{summarise()} ``collapses'' the data to one row per individual
(specified by \texttt{group\_by}).

\begin{Shaded}
\begin{Highlighting}[]
\KeywordTok{summarise}\NormalTok{(wildschwein, }\DataTypeTok{mean =} \KeywordTok{mean}\NormalTok{(timelag, }\DataTypeTok{na.rm =}\NormalTok{ T))}
\end{Highlighting}
\end{Shaded}

\begin{verbatim}
## # A tibble: 2 x 2
##   TierID  mean
##   <chr>  <dbl>
## 1 Hans      15
## 2 Klara     15
\end{verbatim}

Note: You can do \texttt{mutate()} and \texttt{summarise()} on
\texttt{sf} objects as well. However, \texttt{summarise()} tries to
coerce all geometries into one object, which can take along time. To
avoid this, use \texttt{st\_drop\_geometry()} before using
\texttt{summarise()}.

\hypertarget{piping}{%
\subsubsection{Piping}\label{piping}}

The code above may be a bit hard to read, since it has so many nested
functions which need to be read from the inside out. In order to make
code readable in a more human-friendly way, we can use the piping
command \texttt{\%\textgreater{}\%} from \texttt{magrittr}, which is
included in \texttt{dplyr} and the \texttt{tidyverse}. The above code
then looks like this:

\begin{Shaded}
\begin{Highlighting}[]
\NormalTok{wildschwein }\OperatorTok\StringTok{                           }\CommentTok{# Take wildschwein...}
\StringTok{  }\KeywordTok{group_by}\NormalTok{(TierID) }\OperatorTok\StringTok{                    }\CommentTok{# ...group it by TierID}
\StringTok{  }\KeywordTok{summarise}\NormalTok{(                              }\CommentTok{# Summarise the data...}
    \DataTypeTok{mean_timelag =} \KeywordTok{mean}\NormalTok{(timelag,}\DataTypeTok{na.rm =}\NormalTok{ T)}\CommentTok{# ...by calculating the mean timelag}
\NormalTok{  )}
\end{Highlighting}
\end{Shaded}

\begin{verbatim}
## # A tibble: 2 x 2
##   TierID mean_timelag
##   <chr>         <dbl>
## 1 Hans             15
## 2 Klara            15
\end{verbatim}

\hypertarget{bring-it-all-together}{%
\subsubsection{Bring it all
together\ldots{}}\label{bring-it-all-together}}

Here is the same approach with a different dataset:

\begin{Shaded}
\begin{Highlighting}[]
\NormalTok{pigs <-}\StringTok{ }\KeywordTok{tibble}\NormalTok{(}
  \DataTypeTok{TierID =} \KeywordTok{c}\NormalTok{(}\DecValTok{8001}\NormalTok{,}\DecValTok{8003}\NormalTok{,}\DecValTok{8004}\NormalTok{,}\DecValTok{8005}\NormalTok{,}\DecValTok{8800}\NormalTok{,}\DecValTok{8820}\NormalTok{,}\DecValTok{3000}\NormalTok{,}\DecValTok{3001}\NormalTok{,}\DecValTok{3002}\NormalTok{,}\DecValTok{3003}\NormalTok{,}\DecValTok{8330}\NormalTok{,}\DecValTok{7222}\NormalTok{),}
  \DataTypeTok{sex =} \KeywordTok{c}\NormalTok{(}\StringTok{"M"}\NormalTok{,}\StringTok{"M"}\NormalTok{,}\StringTok{"M"}\NormalTok{,}\StringTok{"F"}\NormalTok{,}\StringTok{"M"}\NormalTok{,}\StringTok{"M"}\NormalTok{,}\StringTok{"F"}\NormalTok{,}\StringTok{"F"}\NormalTok{,}\StringTok{"M"}\NormalTok{,}\StringTok{"F"}\NormalTok{,}\StringTok{"M"}\NormalTok{,}\StringTok{"F"}\NormalTok{),}
  \DataTypeTok{age=} \KeywordTok{c}\NormalTok{ (}\StringTok{"A"}\NormalTok{,}\StringTok{"A"}\NormalTok{,}\StringTok{"J"}\NormalTok{,}\StringTok{"A"}\NormalTok{,}\StringTok{"J"}\NormalTok{,}\StringTok{"J"}\NormalTok{,}\StringTok{"J"}\NormalTok{,}\StringTok{"A"}\NormalTok{,}\StringTok{"J"}\NormalTok{,}\StringTok{"J"}\NormalTok{,}\StringTok{"A"}\NormalTok{,}\StringTok{"A"}\NormalTok{),}
  \DataTypeTok{weight =} \KeywordTok{c}\NormalTok{(}\FloatTok{50.755}\NormalTok{,}\FloatTok{43.409}\NormalTok{,}\FloatTok{12.000}\NormalTok{,}\FloatTok{16.787}\NormalTok{,}\FloatTok{20.987}\NormalTok{,}\FloatTok{25.765}\NormalTok{,}\FloatTok{22.0122}\NormalTok{,}\FloatTok{21.343}\NormalTok{,}\FloatTok{12.532}\NormalTok{,}\FloatTok{54.32}\NormalTok{,}\FloatTok{11.027}\NormalTok{,}\FloatTok{88.08}\NormalTok{)}
\NormalTok{)}

\NormalTok{pigs}
\end{Highlighting}
\end{Shaded}

\begin{verbatim}
## # A tibble: 12 x 4
##    TierID sex   age   weight
##     <dbl> <chr> <chr>  <dbl>
##  1   8001 M     A       50.8
##  2   8003 M     A       43.4
##  3   8004 M     J       12  
##  4   8005 F     A       16.8
##  5   8800 M     J       21.0
##  6   8820 M     J       25.8
##  7   3000 F     J       22.0
##  8   3001 F     A       21.3
##  9   3002 M     J       12.5
## 10   3003 F     J       54.3
## 11   8330 M     A       11.0
## 12   7222 F     A       88.1
\end{verbatim}

\begin{Shaded}
\begin{Highlighting}[]
\NormalTok{pigs }\OperatorTok
\StringTok{    }\KeywordTok{summarise}\NormalTok{(         }
    \DataTypeTok{mean_weight =} \KeywordTok{mean}\NormalTok{(weight)}
\NormalTok{  )}
\end{Highlighting}
\end{Shaded}

\begin{verbatim}
## # A tibble: 1 x 1
##   mean_weight
##         <dbl>
## 1        31.6
\end{verbatim}

\begin{Shaded}
\begin{Highlighting}[]
\NormalTok{pigs }\OperatorTok
\StringTok{  }\KeywordTok{group_by}\NormalTok{(sex) }\OperatorTok
\StringTok{  }\KeywordTok{summarise}\NormalTok{(         }
    \DataTypeTok{mean_weight =} \KeywordTok{mean}\NormalTok{(weight)}
\NormalTok{  )}
\end{Highlighting}
\end{Shaded}

\begin{verbatim}
## # A tibble: 2 x 2
##   sex   mean_weight
##   <chr>       <dbl>
## 1 F            40.5
## 2 M            25.2
\end{verbatim}

\begin{Shaded}
\begin{Highlighting}[]
\NormalTok{pigs }\OperatorTok
\StringTok{  }\KeywordTok{group_by}\NormalTok{(sex,age) }\OperatorTok
\StringTok{  }\KeywordTok{summarise}\NormalTok{(         }
    \DataTypeTok{mean_weight =} \KeywordTok{mean}\NormalTok{(weight)}
\NormalTok{  )}
\end{Highlighting}
\end{Shaded}

\begin{verbatim}
## `summarise()` has grouped output by 'sex'. You can override using the `.groups`
## argument.
\end{verbatim}

\begin{verbatim}
## # A tibble: 4 x 3
## # Groups:   sex [2]
##   sex   age   mean_weight
##   <chr> <chr>       <dbl>
## 1 F     A            42.1
## 2 F     J            38.2
## 3 M     A            35.1
## 4 M     J            17.8
\end{verbatim}

\end{document}
